\documentclass[12pt,twoside,letterpaper]{article}

%%%%%%%%%%%%%%%%%%%%%%%%%%%%%%%%%%%%%%%
% include packages necessary for both 
% LaTex and PDFLaTex 
%%%%%%%%%%%%%%%%%%%%%%%%%%%%%%%%%%%%%%%
\topmargin -0.25cm
\textwidth 15.5cm
\textheight 22cm
\oddsidemargin 0.5cm
\evensidemargin 0.5cm

\usepackage{journals}

\usepackage{color}
\usepackage{eso-pic} 
%%  \AddToShipoutPicture{\resizebox{0.9\paperwidth}{0.9\paperheight}%
%%  {\rotatebox{60}{\color[gray]{0.9}\hspace*{5mm}\textsc{Draft \today}}}}

\pagestyle{headings}

% create a new 'if' command: ifpdf
\newif\ifpdf
 \ifx\pdfoutput\undefined
 \pdffalse                  % we are not running PDFLaTeX 
\else 
 \pdfoutput=1               % we are running PDFLaTeX
 \pdftrue
\fi

\ifpdf
%%%%%%%%%%%%%%%%%%%%%%%%%%%%%%%%%%%%%%%
%  include packages necessary for
%  PDFLaTex ONLY 
%%%%%%%%%%%%%%%%%%%%%%%%%%%%%%%%%%%%%%%
 \pdfcompresslevel=0
 \usepackage{url,boxedminipage}
 \usepackage{graphicx,thumbpdf}
 \definecolor{rltred}{rgb}{0.75,0,0}
 \definecolor{rltgreen}{rgb}{0,0.5,0}
 \definecolor{rltblue}{rgb}{0,0,0.75}
 \usepackage[colorlinks% 
  ,bookmarks%
  ,urlcolor=rltblue%
  ,filecolor=rltgreen% 
  ,linkcolor=rltred%
  ,pdftitle={pdftitLe}%
  ,pdfsubject={pdfsubject}%
  ,pdfkeywords={pdfkeywords}%
  ,pdfauthor={Charles Plager <cplager+tdl@fnal.gov>, 
     David Saltzberg <saltzbrg@physics.ucla.edu>}%
  ,pdfpagemode={UseOutlines}%
  ,bookmarksopen=true%
  ,bookmarksnumbered=true%
  ,pdfstartview={Fit}%
  ]{hyperref}
 \pdfimageresolution=300                                
 \DeclareGraphicsExtensions{.pdf,.jpg,.jpeg}            
%
\else
%
%%%%%%%%%%%%%%%%%%%%%%%%%%%%%%%%%%%%%%%
%  include packages necessary for
%  latex ONLY (PS output file)
%%%%%%%%%%%%%%%%%%%%%%%%%%%%%%%%%%%%%%%
 \usepackage{graphicx}                                  
 \DeclareGraphicsExtensions{.eps,.ps,.eps.gz,.ps.gz}    
 \newcommand{\href}[2]{#2}                   % suppress URL's in PS files
 \usepackage[dvips,pagebackref]{hyperref}    % also create Page-back-refrences
                                             % in PS files

\fi % end ifpdf

\DeclareGraphicsRule{*}{mps}{*}{} 

\setcounter{topnumber}{9}
\setcounter{bottomnumber}{9}
\setcounter{totalnumber}{9}
 
\renewcommand{\topfraction}{0.99}
\renewcommand{\bottomfraction}{0.99}
\renewcommand{\textfraction}{0.0}
\renewcommand{\floatpagefraction}{0.99}

% ======================================================================
% ======================================================================
\begin{document}

\thispagestyle{empty}
\vspace*{-3.5cm}
\begin{flushright}
Draft \input{version_number}\\
\today
\end{flushright}

\vspace{0.5in}

\begin{center}
  \begin{Large}
  {\bf A {\em Quick-n-Dirty} BLUE Summary}
  \end{Large}
\end{center}

\vspace{0.5in}

\begin{center}

\href{mailto:cplager+tdl@fnal.gov}{Charles
Plager}\footnote{cplager+blue@fnal.gov},
\vspace{1.0ex} 
\emph{UCLA}
\end{center}

\begin{abstract}
I give a quick summary of BLUE (Best Linear Unbiased Estimator) that
should be sufficient both as a knowledge base to write a note for an
analysis that uses BLUE as well as sufficient to write your own BLUE
implementation.  I finish with a quick discution about the need to
iterate BLUE to avoid bias.
\end{abstract}

%\tableofcontents


% ======================================================================
\section{Introduction}


\hspace*{12pt} First, you can find out about how to get and use my
implementation of blue ({\em CLPBlue})
at\\
\href{http://cdfkits.fnal.gov/SamCode/source/cplager/CLPBlue/README.txt?v=Plager}{http://cdfkits.fnal.gov/SamCode/source/cplager/CLPBlue/README.txt?v=Plager}. \\
BLUE has been described many times in the literature \cite{blue1,
  blue2}.


Before summarizing how BLUE works, let's consider the simple case of
combining $N$ measurements of the same parameter $\mu$ that are
completely uncorrelated with each other.  If we want to find the
average of these measurements, we find the value of $\mu$ that is the
minimum of the $\chi^2$ expression.  The error can also be {\em
  extracted} from the $\chi^2$.  The $\chi^2$ that most of us are
familiar with is

\begin{equation}
\label{eqn:chi2_1}
\chi^2(\mu) = \sum_i^N \left( \frac{ m_i - \mu } {\sigma_i} \right)^2
\end{equation}

where $m_i$ and $\sigma_i$ are the measured value and error of the
{\em i}th analysis, and $\mu$ is the average we are trying to find.


We can also write equation \ref{eqn:chi2_1} as

\begin{equation}
\label{eqn:chi2_2}
\chi^2(\mu) = {\bf \delta}^T(\mu) \cdot {\bf S}^{-1} \cdot {\bf \delta} (\mu)
\end{equation}

where $\delta(\mu)$ is the column vector where $\delta_i = m_i - \mu$
and ${\bf S}$ is the covariance matrix.  In this case, ${\bf S}$ is
diagonal with ${\bf S}_{(i,i)} = \sigma_i^2$.

\section {BLUE}

BLUE is based upon the same formula as equation \ref{eqn:chi2_2} and
takes into account different types of errors and different
correlations between these errors.  If we want to combine $N$
measurements which each have $E$ different types of errors, we have
${\bf S}$ defined as

\begin{equation}
{\bf S} = \sum_{e}^E {\bf S}_e
\end{equation}

and 
\begin{equation}
{{\bf S}_e}_{(i,j)} = {\sigma_e}_i \cdot {\sigma_e}_j 
\cdot {\rho_e}_{(i,j)}
\end{equation}

where ${\sigma_e}_i$ is the $e$th type of error on the $i$th
  measurement and ${\rho_e}_{(i,j)}$ is the correlation of the $e$th
  error between the $i$th and $j$th measurement.

The beauty of BLUE is that you don't have to run a minimization
routine on this $\chi^2$ to get the mean.  Defining ${\bf H}$ to be
the inverse of the covariance matrix and $sum_H$ as below.

\begin{equation}
{\bf H} \equiv {\bf S}^{-1}
\end{equation}

and 

\begin{equation}
sum_H \equiv \sum_{i=1}^N \sum_{j=1}^N {\bf H}_{i,j}
\end{equation}

We then calculate a weight for each measurement $w_i$.

\begin{equation}
w_i \equiv \sum_{j=1}^N \frac {{\bf H}_{i,j}}{sum_H}\\
\end{equation}

Note that weights can be negative, but

\begin{equation}
\sum_{i=1}^N w_i = 1
\end{equation}

BLUE tells us that 

\begin{equation}
\mu_\textrm{best} = \sum_{i=1}^N w_i \cdot m_i
\end{equation}

is the minimum of the $\chi^2$.  The total error $\sigma$ is
calculated as below.

\begin{equation}
\sigma = \frac{1}{\sqrt{sum_H}}
\end{equation}

There are two other useful quantities that BLUE can tell you.  The
first is the total correlations between two measurements given all
error categories

\begin{equation}
\rho_{ij} = \frac {{\bf S}_{(i,j)}} 
    { \sqrt {{\bf S}_{(i,i)} \cdot {\bf S}_{(j,j)}} }
\end{equation}

The second is the breakdown of the error due to the $e$th error
category on the final answer

\begin{equation}
\textrm{Error Breakdown}_e = 
\sqrt{
\sum_{i=1}^N \sum_{j=1}^N w_i \cdot {\sigma_e}_i
\cdot w_j \cdot {\sigma_e}_j \cdot {\rho_e}_{(i,j)}
}
\end{equation}

These last two quantities are now available from {\em CLPBlue}.

\section {Iterative BLUE}

\hspace*{12pt} BLUE is basically a weighted average that takes
correlations into account.  If the errors on each measurement do {\bf
  not} depend on the value that is measured, then the above summary
should be sufficient.  If, however, you do find that smaller
measurements tend to have smaller errors, you will find that simply
running BLUE will produce a bias towards smaller measurements.

To avoid this problem, we need to two steps.  First, we need to get a
functional form of each error in terms of the variable we are
measuring.  Second, we need to make sure that we evaluate the errors
at the value of the final estimation.  To do this, we iterate.

\begin {enumerate}

  \item Take a weighted average assuming no correlations between
    measurements and using all errors as they are.

  \item \label{item:blue} Use BLUE to calculate the mean.

  \item Using the mean value obtained at step \ref{item:blue},
    recalculate all errors.

  \item Repeat until the mean used to calculate the errors and the
    calculation are the same (within what ever small value you desire)

\end {enumerate}

{\bf Note:} My version of blue, {\em CLPBlue} is designed with this
iteration in mind and can do it for you.


\begin{thebibliography}{}

\bibitem{blue1} L.Lyons, D. Gibaut, and P. Clifford, NIM {\bf A270}, 110-117 (1988)
\bibitem{blue2} L.Lyons, A. Martin, and D. Saxon, Phys. Rev. D {\bf 41}, 3 (1990)

\end{thebibliography}

\end{document}

