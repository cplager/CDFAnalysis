\section{Data Sample and Base Event Selection}
For this analysis, we use data collected with the CDF II detector
between March 2002 and September 2006, which corresponds to an
integrated luminosity of $1.12\invfb$. We exclude runs 179057--186598
(``compromised COT'' period) and runs prior to 150145 for muons in the
CMX detector (no triggering on CMX possible).  The data we analyzed
for this analysis was collected with inclusive lepton triggers that
require $\et > 18\gev$ for electrons (datasets \texttt{bhel0d},
\texttt{bhel0h}, and \texttt{bhel0i}) and $\pt > 18\gevc$ for muons
(datasets \texttt{bhmu0d}, \texttt{bhmu0h}, and \texttt{bhmu0i}). We
apply CDF's Good Run List version~16~\cite{GoodRun16},
and require that the data from the silicon detector, electromagnetic
calorimetry, and muon chambers to be marked ``good''. In addition to
data marked good from the muon chambers, we also remove the runs in
which the CMX detector is ignored.
%We also remove the runs without the CMX detector ourselves.

We begin by selecting events with two leptons which form a $Z$, and
four or more jets. We then separate the sample into events with a
loose SECVTX \btag and anti-tagged events. We then further optimize the
selection criteria to obtain the best expected limit and place further
requirements upon the jets \et, transverse mass \mt, and a mass
$\chi^2$.

\subsection{Selection Criteria}
For the base event selection we require exactly one lepton pair of the
same flavor and opposite charge. For \Zee, one of the leptons is a tight 
central electron and the other is a tight central electron, a tight phoenix
electron, or a tight track lepton.  For \Zmm, one of the leptons is a
tight central muon, and the other is a tight central muon or a track
lepton.  The difference in the $z$-coordinate of the point of closest
approach to the beam line, $\Delta z_0$, of the two legs is required to be
less than $5\unit{cm}$, and the dilepton invariant mass is required
to be between 76\gevcsq and 106\gevcsq.

\subsection{Lepton Selection}
This analysis will use tight central electrons, tight
phoenix electrons, and tight central (CMUP or CMX) muons which follow the 
standard CDF Joint Physics criteria for electrons~\cite{JPElectron} and 
muons~\cite{JPMuon}. The selection criteria are listed in 
Tables~\ref{table:TCE}, \ref{table:PHX}, and \ref{table:muons}.
For the selection of tight track leptons, we use the requirements determined for
the lepton+track top \xsect measurement~\cite{CDF8696}, as shown in
Table~\ref{table:tracks}. In addition, we impose a correction for 
electron+track $Z$ candidates to account for possible energy loss due to 
bremsstrahlung. In this case, a collinear photon will deposit energy into a 
nearby calorimetry tower. We can use the energy from the photon to recover 
the energy loss of the track. For these tracks, we use \et if it is larger than \pt.


\subsection{Jet Selection}
In addition to the $Z$ candidate, we require 4 or more jets with
$\et> 15\gev$ corrected to jet energy correction level 5 and $|\eta|
\leq 2.4$. In events with five or more jets, only the four leading
jets are considered. Events with $Z+\geq 4$ jets fall into our signal region and
are therefore blinded. We use the events with a $Z$ and three or fewer
jets as our control region to perform cross-checks for our background
prediction methods. For some studies, we will shrink the blind signal
region by applying a cut on a mass $\chi^2$ variable.
 
\subsection{Additional Selection Criteria}
After the base selection, we apply further cuts to improve the
background rejection. These cuts are optimized for the best expected
limit. We found that a ``sliding'' cut on the transverse energy of the
four leading jets, a cut on the transverse mass of the system, \mt, and a
mass $\chi^2$ are the most sensitive. The definitions of these variables, 
calculation of the expected limit, and the subsequent optimization of 
the additional selection criteria are described in 
Section~\ref{section:limitsandoptimization}.

\subsubsection{``Sliding'' Jet Transverse Energy}
The FCNC signal comes from the decay of massive top quarks and
contains four hard jets. Therefore a ``sliding'' cut on the transverse
energies of the four leading jets in an event suppresses background
from \Zj events that originates from a lower energy process and from
SM \ttbar, where two of the four jets have to come from initial or
final state radiation. 
%We found a cut of $\et>40\gev$ for the leading
%jet, $\et>30\gev$ for the second leading jet, $\et>20\gev$ for the
%third leading jet, and $\et>15\gev$ for the fourth leading jet to be
%optimal for both the anti-$b$-tagged and the $b$-tagged analysis.

\subsubsection{Transverse Mass}
A second variable with good discriminating power is the
transverse mass of the full event, defined as
\begin{equation}
  \mt = \sqrt{ \left(\sum\et \right)^2 - \left(\sum\vec{p}_{T}\right)^2 },
\end{equation}
where the sums include the four leading jets and the reconstructed $Z$. 

\subsubsection[Mass $\chi^2$]{Mass \boldmath $\chi^2$\unboldmath}
\label{section:masschi2}
We make use of the ability to fully reconstruct the event kinematics to
separate signal from background by constructing a mass $\chi^2$
variable. In a signal event, there is one decay of the type \tWb.
Two jets in the event form a $W$, which in turn form a top quark together with
a third jet. There is also one decay of the type \tZq, in which the
$Z$ has to be paired with the fourth jet to form the second top. The mass
$\chi^2$ is defined as

\begin{equation}
\chi^2 = 
\left(\frac{m_{W,\mathrm{rec}} - m_{W,\mathrm{PDG}}}{\sigma_{W, rec}} \right)^2 +
\left(\frac{m_{\tWb,\mathrm{rec}} - m_{t,\mathrm{PDG}}}{\sigma_{\tWb}} \right)^2 +
\left(\frac{m_{\tZq,\mathrm{rec}} - m_{t,\mathrm{PDG}}}{\sigma_{\tZq}} \right)^2,
\end{equation}

where, for each permutation, the masses are obtained by the following prescription:

\begin{enumerate}
\item Correct the four leading jets four-vectors with level-5 jet energy scale
  corrections and top-specific corrections taken from the Top Mass
  Template (TMT) analysis~\cite{CDF7532}.
\item Calculate the invariant mass of the first two jets to form the mass
  of the reconstructed $W$, $m_{W,\mathrm{rec}}$.
\item Vary the momentum four-vectors of both $W$ daughters within their respective
  resolutions such that the $W$ mass is fixed to its PDG value. The
  resolutions are taken from the TMT analysis.
\item Calculate the invariant mass of the $W$ and the third jet to form the reconstructed
  top mass, $m_{\tWb,\mathrm{rec}}$.
\item Reconstruct a $Z$ from two leptons.
\item Fix the $Z$ mass to its PDG value by varying the two lepton four-vectors. We 
  assume the lepton's resolutions are a constant percentage of their total momenta.
\item Calculate the invariant mass of the $Z$ and the fourth jet, 
  $m_{\tZq,\mathrm{rec}}$.
\end{enumerate}

In the above calculation for the mass $\chi^{2}$ distribution, we assume a top mass
of 175\gevcsq. The widths shown are the RMS values of the HEPG-matched reconstructed
signal Monte Carlo, where $\sigma_{W, rec}$, $\sigma_{\tWb}$, and $\sigma_{\tZq}$
are 15\unit{GeV}, 24\unit{GeV}, and 21\unit{GeV}, respectively. The above prescription 
is repeated for all possible permutations of the four leading jets in the event, and 
the permutation with the lowest $\chi^2$ is selected. We do not make use of $b$-tagging 
information in resolving the combinatorics.

%: $\sqrt{\chi^2} < 1.35$ for anti-tagged
%events, and $\sqrt{\chi^2} < 1.6$ for loose SECVTX tagged events. 

\subsubsection{Lepton+Jets Selection}
Our expected number of signal events are normalized to the lepton+jets SECVTX
\xsect. The lepton+jets analysis searches in \ttbar pairs for the standard model 
top decay \tWb, where one top decays hadronically and the other leptonically 
yielding the lepton + three or more jets + missing \Et event signature. They require one 
tight lepton (TCE, CMUP, CMX), three jets with $\Et>20\gev$ corrected to jet 
energy correction level 5, and at least two loose SECVTX $b$-tags. Details of their acceptance 
selection criteria are given in Table~\ref{table:LepJets}.

%After optimizing our event selection for the best expected limit, we
%make further requirements for our events. We require higher transverse
%energy for the highest four jets to be greater than 50\gev, 40\gev,
%30\gev, and 20\gev, respectively. We also require the scalar sum of
%the \et s (\pt s) of the leptons and the \et s of the jets to be above
%260\gev. We also require the mass $\chi^2$, which measures the
%deviations of the masses of the reconstructed FCNC top, W, and SM top
%from the average in matched Monte Carlo events, to be less than
%5x.x.

