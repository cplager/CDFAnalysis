\section{Method}
We perform a blind search for the FCNC decay \tZq, with the signal
region defined as events with a reconstructed $Z$ in the mass range of
76--106\gevcsq, four or more jets, and mass \chisq (constructed from
reconstructed $W$, SM top, and FCNC top masses) less than 9. We split
the data sample into two subsamples that are analyzed separately, a
loose SECVTX $b$-tagged and an anti-$b$-tagged sample. We take the signal
acceptances and efficiencies from a Monte Carlo (MC) simulation with
appropriate scale factors and trigger efficiencies applied. We then 
normalize the expected number of signal events to the lepton+jets
top cross section. The dominant background for the signal
region comes from \sm \Zj production, in which we estimate using a rigorous 
combination of data and Monte Carlo techniques. Other background contributions 
include SM top and dibosons, estimated using MC simulation, and \Wj production,
estimated using both data and MC simulation.

The event selection for the $b$-tagged and the anti-$b$-tagged data samples
are optimized for the best combined expected limit. We use both
Feldman-Cousins (FC) and Bayesian frameworks for calculating expected
limits. Both frameworks take systematic uncertainties into account 
in the limit calculation. We find that the limits obtained in either 
framework track each other well. As the limit calculation in the FC 
framework is very CPU-intensive, we optimize our selection criteria 
for the best expected limit using the Bayesian framework and use the 
FC framework to obtain the limit for our signal event yields.  After
the final selection criteria have been chosen, we derive a limit on
the branching fraction of the decay \tZq from the number of events
observed in the signal region and the number of expected background
events.
