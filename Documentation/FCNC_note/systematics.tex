\section{Systematic Uncertainties}
\label{section:systematics}

We have checked a variety of systematic uncertainties both for the
FCNC signal and for the most important source of background events,
the \Zj production. We have also studies the systematics associated 
with the normalization to the lepton+jets \xsect.  The results of our 
studies are summarized in Table~\ref{table:signal_systematics_summary} for the signal
systematics and Table~\ref{table:zjets_systematics_summary} for the
\Zj background systematics.

\subsection{Signal Systematics}

\subsubsection{Luminosity}
As we are normalizing the FCNC acceptance to the acceptance calculated for
the top cross section analysis, the common factor of the luminosity uncertainty
drops out in the ratio.

\subsubsection{Monte Carlo Statistics}
All the MC samples used for this analysis contain a sufficiently large
number of events, so that the uncertainty coming from MC statistics is
reflected by the small statistical uncertainties of the acceptances, and
are therefore negligible.

\subsubsection{\tZq Helicity Reweighing}
\label{section:helicitysystematics}

In order to study the effect of the $Z$ helicity on the FCNC signal
acceptance we have weighted our original MC sample to obtain different
admixtures of longitudinal, left-handed, and right-handed components,
and measured the acceptances for the re-weighted
samples. As generated, the Monte Carlo does not produce a perfectly flat
\costh distribution, as shown in Fig.~\ref{fig:costheta}. Before re-weighting
to the appropriate 65\% longitudinal and 35\% left-handed helicity, we correct
for the approximate flatness by fitting a straight line to the distribution and 
applying a correction factor. Shown  in Fig.~\ref{fig:costheta}b, the original 
distribution is corrected and appropriately re-weighted to the SM expectation. 
Table~\ref{table:reweighting} gives the acceptances for a
collection of re-weighted samples.  Here acceptance is defined as the
ratio of the number of reconstructed signal MC events with a $Z$ candidate 
and four or more jets to the number of generated signal MC
events. Here, we define events with a $Z$ candidates as events with two tight leptons, 
or one tight lepton and one tight track, with scale factors and trigger efficiencies
included, which form a $Z$ in the 76\gevcsq--106\gevcsq mass window. In general, 
the effect of the unknown helicity on the
acceptance is small. The biggest difference occurs between the extreme
cases of a 100\% longitudinal sample and a 100\% left-handed sample,
with a relative acceptance change of 5\%. Since these two values are absolute extremes, 
we consider half the difference, 2.5\%, as the $2\sigma$ systematic uncertainty on the $Z$ helicity.

\begin{table}
  \caption{Change of raw acceptance in the FCNC signal MC for different assumed $Z$ helicities.}
  \label{table:reweighting}
  \begin{center}
    \small
    \begin{tabular}{lD{;}{\pm}{-1}} 
      \toprule
      {\bf Helicity}      & \multicolumn{1}{c}{\bf Acceptance (\%)}  \\ 
      \midrule
      Flat & 17.43;0.05\\
      100\% Longitudinal & 16.93;0.06\\
      100\% Right-Handed & 17.61;0.07\\
      100\% Left-Handed & 17.77;0.07\\
      35\% Left-Handed, 65\% Longitudinal & 17.22;0.06\\
      \bottomrule
    \end{tabular}
  \end{center}
\end{table}

%Another concern for the flat \costh sample is that the \pt shapes of
%the leptons and other distributions will be different than for the
%sample with the correct helicity\footnote{%
%  Strictly speaking, it is the same concern as above, because
%  reweighting the underlying variable \costh corrects the lepton \pt
%%%%%  spectra as well.}.
%We have studied this effect by generating \ttbar events in which one
%top quark decays to a $W$ and a $b$ quark and the other decays to a
%charged Higgs boson $H^+$ and a $b$ quark where the $H^+$ has the same
%mass as $W$ and a vanishing width. Since $t \rightarrow H^+ b$ is not
%a weak decay vertex, \pyth decays the top isotropically instead of
%with a $V\!-\!A$ vertex. We forced the $H^+$ to decay to an electron
%(muon) and a neutrino, and the $W$ to decay to two quarks. We compared
%these events to SM $\ttbar \rightarrow W^+ b\, W^- \bbar$ events in
%which one $W$ decays leptonically and the other decays
%hadronically. Fig.~\ref{fig:higgscheck}a shows the \costh
%distribution of the $t \rightarrow H^+ b$ decay, which is
%flat. Fig.~\ref{fig:higgscheck}b shows the \pt of the lepton from
%the $H^+$ decay compared to the \pt of the lepton from the W decay in
%the SM sample. Fig.~\ref{fig:higgscheck}c shows the \pt of the
%lepton from the $H^+$ decay after the sample has been reweighted to
%have a 70\% longitudinal and 30\% left-handed components compared to
%the \pt of the lepton from the $W$ decay in the Standard Model
%sample. After reweighting the \pt distributions are the same,
%indicating that reweighting the FCNC signal sample will give us the
%correct distributions for \pt of the leptons and other variables.

%\begin{figure}[t]
%\subfigure{}
%\subfigure{}
%\caption{Comparison of kinematic distributions for $t\to H^+ b$ and
%  $t\to W b$: a) \costh for $t\to H^+ b$, b) unweighted lepton \pt, c)
%  weighted lepton \pt.}
%\label{fig:higgscheck}
%\end{figure}

%To estimate the systematic effect of the model-dependent admixture of
%left-handed and right-handed components, we calculate the signal
%acceptance with a fixed longitudinal component of 65\% for left-handed
%to right-handed ratios of 35\%:0\%, 0\%:35\%, and 17.5\%:17.5\%. In
%addition we check the extreme cases of a 100\% left-handed and a 100\%
%right-handed decay. 


\subsubsection{Initial and Final State Radiation}
The modeling of initial state radiation (ISR) and final state
radiation (FSR) in the MC simulation is expected to have an effect on
the number of reconstructed jets. As the signal acceptance requires
four or more reconstructed jets, it is influenced by the amount of ISR
and FSR.  We have generated MC events to study the effect of the
modeling of ISR and FSR on the FCNC signal acceptance. The generated sample
includes 50,000 signal events ( $Z(ll)W(q\overline{q}')$)
each for the scenarios of more ISR, less ISR, more FSR, and less
FSR. The \pyth settings for these samples are those used for similar
studies of the systematics of \ttbar production in Gen5.  The
settings are summarized in Table~\ref{table:isrfsr}. The relevant
\pyth parameters are~\cite{Sjostrand:2003wg}:

\begin{itemize}
\item \texttt{PARP(1)}: nominal value of \LambdaQCD for the running of
\alphas (default: 0.25, units: GeV).
\item \texttt{PARP(61)}: value of \LambdaQCD for the running of
\alphas in space-like shower evolution (default: 0.25, units: GeV).
\item \texttt{PARP(72)}: value of \LambdaQCD for the running of
\alphas in time-like shower evolution (default: 0.25, units: GeV).
\item \texttt{PARP(64)}: $Q^2$ scale times this factor is maximum
parton virtuality in space-like showers (default: 1).
\item \texttt{PARP(71)}: $Q^2$ scale times this factor is maximum
parton virtuality in time-like showers (default: 4).
\end{itemize}


\begin{table}[t]
\small
  \begin{center}
    \caption{\pyth settings for the MC samples to study systematic
      effects due to the amount of initial and final state
      radiation. The \pyth parameters are described in the text.}
    \label{table:isrfsr}
    \vspace{2mm}

  \begin{tabular}{lccccc}
    \toprule
    {\bf Setting} & 
    {\bf PARP(1)}& 
    {\bf PARP(61)}& 
    {\bf PARP(72)}& 
    {\bf PARP(64)} & 
    {\bf PARP(71)} \\
    \midrule
    More ISR & 0.146 & 0.292 & 0.146 & 0.5 & 4.0 \\
    Less ISR & 0.146 & 0.072 & 0.146 & 2.0 & 4.0 \\
    More FSR & 0.146 & 0.146 & 0.292 & 1.0 & 8.0 \\
    Less FSR & 0.146 & 0.146 & 0.076 & 1.0 & 2.0 \\
    \bottomrule
  \end{tabular}
  \end{center}
\end{table}

%NEED ACCEPTANCE NUMBERS, ANY CHANGE IN CHI2 PERFORMANCE?
We have studied the systematic uncertainties due to more or less
ISR/FSR by computing the relative change in acceptance for the
more/less ISR/FSR samples, as compared to 50,000 events from the FCNC
signal MC sample after helicity re-weighting, covering the same run
range. We did not observe a significant change in any kinematic
variable other than the jet multiplicity. The results are summarized
in Table~\ref{table:systematics_isrfsr}. We assign half of the
difference between the largest and the smallest acceptance as the
systematic uncertainty, 0.013.


\begin{table}[t]
  \begin{center}
    \caption{\pyth settings for the MC samples to study systematic
      effects due to the amount of initial and final state
      radiation. The \pyth parameters are described in the text.}
    \label{table:systematics_isrfsr}
    \vspace{2mm}

  \begin{tabular}{lD{;}{\pm}{-1}D{.}{.}{-1}}
    \toprule
    {\bf Setting} & 
%    \multicolumn{1}{c}{\bf Acceptance} &
    \multicolumn{1}{c}{\bf Relative Change} \\
    \midrule
    Default   & 0.0 \\
    \midrule
    More ISR  & +0.001 \\
    Less ISR  & -0.016 \\
    More FSR  & +0.009 \\
    Less FSR  & +0.005 \\
    \bottomrule
  \end{tabular}
  \end{center}
\end{table}

\subsubsection[$B$-Tagging Scale Factor]{\boldmath $B$-Tagging Scale Factor\unboldmath}


%\subsubsection{Parton Distribution Functions}

%\subsubsection{Jet Energy Scale}
%We calculate systematic uncertainties due to the jet energy scale as
%described on the Jet Energy Corrections Web Page~\cite{JES}. We vary
%the parameters of the level-5 jet energy corrections by $\pm 1 \sigma$
%and study the effect on the signal acceptance.

%\subsubsection{Lepton ID}
%We take the systematic uncertainties recommended by the Joint Physics group\dots

\subsection{Background Systematics}

\subsubsection{\alp and MLM Matching}

The \alp MC generator allows to vary several parameters which
influence the MLM matching algorithm and the factorization and
renormalization scale $Q^2$. The central value for the parameters have
been determined partly based on advice by the authors of \alp, partly
by validating \alp against CDF data outside the FCNC signal region. We
have generated test samples for which we varied the \alp settings,
containing 50,000 events each for $Z+0,1,2,3,4$\,partons,
$Z+\ccbar+0,1,2$\,partons, and $Z+\bbbar+0,1,2$\,partons, where the
$Z$ decays into \mm pairs only.


\begin{itemize}
\item{\bf Matching of parton \pt and cluster \et}: \alp is a MC
  generator that handles the production of jets both from hard matrix
  elements and \pyth parton showers. \alp allows to set the
  energy/momentum scale above which jets are generated via matrix
  elements and below which parton showers take over. There are
  separate scales for the parton \pt and the cluster \et, but they are
  commonly taken to be the same, with a default value of
  $(\et,\pt)=(15\gev,15\gevc)$. We have studied samples with
  $(10\gev,10\gevc)$ and $(20\gev,20\gevc)$. Changing the matching cut
  changes the relative \xsects of the $n$-parton subsamples by as much
  as a factor of 20; however, when the samples are combined according
  to their relative cross sections, the effect on the total cross
  section is small, from $-0.012$ for $(10\gev,10\gevc)$ to $+0.015$
  for$(20\gev,20\gevc)$. 
\item {\bf \boldmath $Q^{2}$ \unboldmath scale for parton distribution
    functions}: In \alp, the renormalization and factorization scale
  can be picked.
\item {\bf \boldmath $Q^{2}$ \unboldmath scale for each vertex}: In
  \alp, the value of the QCD coupling $\alphas(Q^2)$ is evaluated at
  every strong vertex individually. There are two options for the
  choice of the energy scale $Q^2=\pt^{~2}$, and $Q^2=m_T^{~2}$, where
  \pt and $m_T$ are the transverse momentum and the transverse mass of
  the vertex. There is an additional parameter to vary the scale of
  $\alphas(Q^2)$ around its central value. We have generated samples
  for both $\pt$ and $m_T$ with scale factors of $0.5$, $1.0$, and
  $2.0$. 
\end{itemize}
To be finalized\dots

%\subsubsection{Heavy Flavor Fractions}


%\subsubsection{Jet Energy Scale}

%\subsubsection{Initial and Final State Radiation}

\subsubsection[$B$-Tagging Scale Factor]{\boldmath $B$-Tagging Scale Factor\unboldmath}
	
%\subsubsection{Lepton ID}

%\subsubsection{Monte Carlo Statistics}

\subsubsection{Luminosity}
Only the diboson background receives its absolute normalization from the integrated luminosity, 
for which we assign a systematic uncertainty of 6\%.

 
\subsection{Systematics Summary}

\begin{table}
  \begin{center}
    \caption{\label{table:signal_systematics_summary} Summary of all systematic
      uncertainties on the signal.}
    \vspace{2mm}

    \small
    \begin{tabular}{cD{;}{\pm}{-1}} 
      \toprule
      \multicolumn{1}{c}{\bf Source} & 
      \multicolumn{1}{c}{\bf Uncertainty} \\
      \midrule
      \\
      \midrule
      Total\\
      \bottomrule
    \end{tabular}
  \end{center}
\end{table}

\begin{table}
  \begin{center}
    \caption{\label{table:zjets_systematics_summary} Summary of all systematic
      uncertainties on the background.}
    \vspace{2mm}

    \small
    \begin{tabular}{cD{;}{\pm}{-1}} 
      \toprule
      \multicolumn{1}{c}{\bf Source} & 
      \multicolumn{1}{c}{\bf Uncertainty} \\
      \midrule
      \\
      \midrule
      Total\\
      \bottomrule
    \end{tabular}
  \end{center}
\end{table}
