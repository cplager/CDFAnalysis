\section{Introduction}
In the \sm of particle physics (SM), flavor changing
neutral current (FCNC) decays are highly suppressed. They
do not occur at tree level, and are only 
allowed at the level of quantum loop corrections at very 
small branching fractions, as shown in CDF Note 7744~\cite{CDF7744}. 
The branching fraction for the top quark decay \tZq is predicted 
to be $\mathcal{O}(10^{-14})$, far below the experimental sensitivity 
of the Tevatron or even the Large Hadron Collider (LHC). As summarized 
by J.A.\ Aguilar-Saavedra~\cite{Aguilar-Saavedra:2004wm}, there exist 
new physics models that predict much higher branching fractions, 
up to $\mathcal{O}(10^{-2})$. Any detection of top's FCNC decay at the 
Tevatron would be an indication of new physics. 

We search for \ttbar events in which either top quark decays via 
the FCNC to a $Z$ boson and a quark ($u$ or $c$), and
the other top quark decays via the SM to a $W$ boson and a $b$ quark.  We
examine the decay channel in which the $Z$ subsequently decays to a
pair of charged leptons (\ee or \mm), and the $W$ decays to two
quarks. We also evaluate the case in which both tops decay via FCNC, and
take into account its impact on the signal acceptance. Although the branching 
fraction of $Z\rightarrow\ee/\mm$ is only 3.33\% per channel, 
it is a very clean channel to identify that there was a $Z$ present in the 
decay. We chose the \Wqq decay mode because it has the highest branching 
ratio and contributes to a final state with a large jet multiplicity.  
Thus, the final signature is a reconstructed $Z$ and four or more jets, 
one of which is a $b$-jet that can be identified using a loose secondary vertex 
(SECVTX) \btag. Our experimental signature does not include any neutrinos in the
final state, and we are therefore able to fully reconstruct the event.

The dominant background contribution for this analysis comes from
$Z$s produced in association with jets (\Zj). The other 
background contribution comes from SM $\ttbar\to\Wp b\, \Wm \bbar$
events. In the dilepton decay mode, the invariant mass of two leptons
forms a continuum background to the $Z$ peak, and in the lepton+jets
decay mode, the invariant mass of the lepton and a jet that is
misidentified as a lepton may fall within the Z mass window.  There is
a contribution similar in size from electroweak production of pairs of
gauge bosons, or ``dibosons'', where we consider both $WZ$ and $ZZ$ events. 
The contribution from $W$s produced in association with jets (\Wj) and
the $WW$ diboson production is negligible.
